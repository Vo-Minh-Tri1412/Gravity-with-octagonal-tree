
\section{Bài tập ứng dụng}
\begin{enumerate}[label={\bfseries Bài \arabic*.}]
\item  Cho tập hợp
\[
U = \{1,2,3,4,5,6,7,8\}
\]
và các tập con:
\[
A = \{1,2,4,8\},
\]
\[
B = \{x \in U \mid x \text{ là số lẻ}\},
\]
\[
C = \{5,6,7,8\}.
\]

Tìm các tập hợp hoặc tính các giá trị sau:
\begin{enumerate}[(a)]
    \item \(A \cup B\)
    \item \(A \cap B\)
    \item \(A \setminus B\)
    \item \(A^c\)
    \item \((A \cup B) \setminus C\)
    \item \(A^c \cup C\)
    \item \(A^c \cap C\)
    \item \(|A|,\ |A| - |B|,\ |A \setminus C|\)
\end{enumerate}

\item Có bao nhiêu bội số của 7 trong các số nguyên từ 1 đến 1001.

\item Có bao nhiêu số lẻ từ 100 đến 999 có các chữ số đôi một khác nhau.

\item Một xe lửa có 5 toa ngừng ngang một nhà ga để 3 hành khách bước lên tàu. Hỏi có bao nhiêu cách lên tàu của 3 hành khách này nếu:
\begin{enumerate}
    \item Ai lên toa nào cũng được?
    \item Mỗi người lên một toa khác nhau?
\end{enumerate}

\item Vào thời điểm nước ta có 64 tỉnh thành, hãy chứng tỏ rằng trong một lớp gồm 70 sinh viên Việt Nam có ít nhất 2 sinh viên đồng hương.

\item Một khu vườn có tất cả 10 loại hoa. Chứng minh rằng nếu hái 12 bông trong vườn sẽ có được ít nhất 2 bông hoa cùng loại.

\item Chọn ngẫu nhiên $10$ số tự nhiên bất kỳ. Chứng minh rằng luôn tồn tại một tập hợp con (không rỗng) của $10$ số đó sao cho tổng các phần tử của tập hợp con này chia hết cho $10$.

\item Cho $5$ điểm bất kỳ nằm trong một hình vuông có cạnh dài $2$ đơn vị. Chứng minh rằng luôn tồn tại ít nhất hai điểm trong số đó có khoảng cách không quá $\sqrt{2}$.

\end{enumerate}

\label{page:lastpage}